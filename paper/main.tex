\documentclass[12pt]{ltjsarticle}
\usepackage{luatexja}
\usepackage{bussproofs}
\usepackage{amsmath,amssymb,amsthm}
\usepackage{tikz}
\usetikzlibrary{matrix, positioning}
\begin{document}

\theoremstyle{definition}
\newtheorem{defn}{定義}
\newtheorem{thm}{定理}
\newtheorem{lem}{補題}
\newtheorem{rem}{注意}
\newtheorem{cor}{系}
\newtheorem{ex}{例}
\renewcommand{\proofname}{\bf{証明}}

\title{Calculus of Constructionの基本的性質}
\author{
アドバイザー名: Jacques Garrigue
学生番号: 321701183
氏名: 田中 一成
}
\maketitle

% 構成案
% 
% 型システムを論理体系とみなすために必要な事実とは
% 合流性 (等式の判定のため)
% 正規性 (?, 不等式の場合にかんがえてみるべき?)
% - equiv. canonicity?
% Subject Reductionをいれるか?
% lambda cubeをみせるか?
% いくつかの主張を型で書いてみるとか?
% - False型 (任意の主張がせいりつする)
% Cocに対応するtheoryを構成するべきか?
% Cocのモデルは必要か?
%  - Scottの方法がつかえるか?
% rewritingについてどれだけのべるか?
% - あまり必要ない?
% - completionはheuristicなので説明しづらい
% - Universal Algebraくらいは述べてもいいかな?
%   - theoryの記述につかえる
% 数学的な記述 vs BNF
% Polymorphic typeの整合性
% - 集合論での矛盾 (Gilard)
% - System Fのモチベーション
%   - Prop as Type
% Intuitionistic logicの歴史はいる?
% 
% 適切なReductionを入れることが重要であると私はおもっている...
% - directed homotopy? (わかってないので書かない)
% - よくかんがえたらこの問題意識はcompletion由来かも?

\begin{abstract}
\end{abstract}

% いらないのでは?
\section{記号等}
初めに簡約関係と呼ばれる一般の$2$項関係について論じる. 考えている集合$A$とその上の$2$項関係$\rightarrow \subset A \times A$を固定する.ここでそれぞれ,
$\xrightarrow{n}$を$\rightarrow$を$n$回合成した関係,
$\leftrightarrow$を$\rightarrow$の対称閉包,
$\xrightarrow{+}$を$\rightarrow$の推移閉包,
$\xrightarrow{=}$を$\rightarrow$の反射閉包,
$\overset{*}{\rightarrow}$を$\rightarrow$の推移反射閉包,
$\overset{+}{\leftrightarrow}$を$\rightarrow$の推移対称閉包,
$\overset{*}{\leftrightarrow}$を$\rightarrow$の推移対称反射閉包とする.
また$(a, b) \in \rightarrow$を$a \rightarrow b$と書く.

\begin{defn}
$\rightarrow$が合流性を持つとは,任意の$y_{1},y_{2},x \in A$について$y_1 \overset{*}{\leftarrow} x \overset{*}{\rightarrow} y_2$ならば, $y_1 \overset{*}{\rightarrow} z \overset{*}{\leftarrow} y_2$となる$z \in A$が存在することを言い, 停止するとは無限列$a_0 \rightarrow a_1 \rightarrow a_2 \rightarrow \cdots$が$A$に存在しないことを言う. さらに$x \in A$が正規形であるとは, $x \rightarrow y$となる$y \in A$が存在しないことを言う.
\end{defn}

また停止性の非自明な言い換えとしてwell foundnessとよばれるものがある.

\begin{defn}
$\rightarrow$がwell foundedであるとは, 以下の条件が同値であることをいう. ただし$P$は任意の述語を表している.
 \begin{itemize}
  \item 任意の$x \in A$について, $x \xrightarrow{+} y$となる全ての$y \in A$に対して$P(y)$ならば$P(x)$
  \item 任意の$z \in A$について$P(z)$
 \end{itemize}
\end{defn}

\begin{thm}
 停止性とwell foundnessは同値である.
\end{thm}

\begin{proof}
 必要性を示す. 任意の$w \in A$について$P(w):= \text{無限列} w \rightarrow w_1 \rightarrow \cdots \text{は存在しない}$と置く. 任意の$x \xrightarrow{+} y$について$P(y)$が成立するとすると, $x$から始まる無限列は起こり得ない. したがって$P(x)$が成立する. $\rightarrow$がwell foundedであることより, 任意の$z \in A$に対して$P(z)$, すなわち$\rightarrow$の停止性が従う.
 
 十分性を背理法により示す. 任意の$x \xrightarrow{+} y$について$P(y)$ならば$P(x)$かつ, ある$z \in A$が存在して$P(z)$が成立しないと仮定する. 前半の主張の対偶をとると, 任意の$x \in A$に対して$P(x)$が成立しないならば, ある$y \in A$で$x \xrightarrow{+} y$かつ$P(y)$が成立しないものが存在することになる. この主張を用いて$z$から始まる無限列を構成することができる.
\end{proof}

\begin{rem}
帰納的に定義された(有限)木構造に対して, 自然な順序が存在する.
このときその順序に対して, well foundnessを常に適用できるが, これを断りなく帰納法と呼ぶ.
\end{rem}

%\section{型理論とは}
% 仰々しい?
% 1

\section{ラムダ計算}\label{lambda}
この節ではCoCの基礎となる(型なし)ラムダ計算のいくつかの事実を述べる.

\begin{defn}
 ラムダ項$t$とは, 以下のBNFで定義される木である. $v$は可算濃度の集合$V$の元を意味している.
 \[
 t := v | \lambda v. t | t t
 \]
\end{defn}

BNFでの定義と同値なものが以下である.

\begin{defn}
 ラムダ項の集合$\Lambda$とは, 以下の条件を満たす最小の集合のことをいう.
 \begin{itemize}
  \item $V \subset \Lambda$
  \item 任意の$v \in V$, $M \in \Lambda$について, $\lambda v. M \in \Lambda$
  \item 任意の$M \in \Lambda$, $N \in \Lambda$について, $M N \in \Lambda$
 \end{itemize}
\end{defn}

ただし, ここではラムダ項そのものの定義ではなくラムダ項の集合を定義している. 以下では記述を簡潔にするため, 構文の定義としてBNFでの表記を用いることとする.

ラムダ計算のシステムは計算として機能するための作用を持つ. そのうちの最も本質的なものが$\beta$簡約と呼ばれている.

\begin{defn}
 $\beta$簡約$\rightarrow_{\beta}$とは$\Lambda$の上の$2$項関係であり, 以下の条件を満たす最小の関係のことをいう.
 \begin{itemize}
  \item 任意の$v \in V$, $M, N \in \Lambda$について, $(\lambda v. M) N \rightarrow_\beta M[N/v]$
  \item 任意の$v \in V$, $N_1, N_2 \in \Lambda$について, $N_1 \rightarrow_\beta N_2$ならば$(\lambda v. N_1) \rightarrow_\beta (\lambda v. N_2)$
  \item 任意の$M, N_1, N_2 \in \Lambda$について, $N_1 \rightarrow_\beta N_2$ならば$M N_1 \rightarrow_\beta M N_2$かつ$N_1 M \rightarrow_\beta N_2 M$
 \end{itemize}
 ただし, ここで$M[N/v]$とは代入と呼ばれる操作を表している. 代入とはラムダ項$M \in \Lambda$の中に表れる$v \in V$をすべてラムダ項$N \in \Lambda$に置きかえるということである.
\end{defn}

ここで述べた代入という操作を正確に記述するために, de brujin indexとよばれるものを導入する.
集合$V$は変数の集合を表しており, 我々の状況では$V := \mathbb{N}$として問題ないため, 以下ではそのようにする.

また, ラムダ項の定義を修正し以下のようにする.

\begin{defn}
 \[
 t := v | \lambda. t | t t
 \]
\end{defn}

\begin{defn}[de brujin index]
 写像 $s \colon \Lambda \times \mathbb{N} \times \mathbb{N} \rightarrow \Lambda$を任意のラムダ項$M$, 自然数$n, c$に対して, 以下のように帰納的に定義する.
 
\[
  s (M, n, c) := \begin{cases}
    M & (M \in V \text{かつ} M < c) \\
    M + n & (M \in V \text{かつ} M \geq c) \\
    \lambda. s (N, n, c + 1) & (M = \lambda. N) \\
    s (N_1, n, c) s (N_2, n, c) & (M = N_1 N_2)
  \end{cases}
\]
 この写像$s$を用いて, ラムダ項$M, N \in \Lambda$, 変数$v \in V$に対する代入を以下のように帰納的に定義する.
 
\[
 M[N/v] := \begin{cases}
    N & (M = v) \\
    M - 1 & (M \in V \text{かつ} v < M) \\
    M & (M \in V \text{かつ} v \geq M) \\
    \lambda. M' [s (N, 1, 0)/(v + 1)] & (M = \lambda. M') \\
    N_1[N/v] N_2[N/v] & (M = N_1 N_2)
  \end{cases}
\]
\end{defn}

$\beta$簡約の定義も修正する.

\begin{defn}
 $\beta$簡約$\rightarrow_{\beta}$とは$\Lambda$の上の$2$項関係であり, 以下の条件を満たす最小の関係のことをいう.
 \begin{itemize}
  \item 任意の$M, N \in \Lambda$について, $(\lambda. M) N \rightarrow_\beta M[N/0]$
  \item 任意の$N_1, N_2 \in \Lambda$について, $N_1 \rightarrow_\beta N_2$ならば$(\lambda. N_1) \rightarrow_\beta (\lambda. N_2)$
  \item 任意の$M, N_1, N_2 \in \Lambda$について, $N_1 \rightarrow_\beta N_2$ならば$M N_1 \rightarrow_\beta M N_2$かつ$N_1 M \rightarrow_\beta N_2 M$
 \end{itemize}
\end{defn}

\begin{ex}[$K$ コンビネータ]
 $V$を文字列の集合とする. このとき $x, y \in V$, $z, w \in \Lambda$に対して以下の様な例が考えられる.
 \[
 (\lambda x. \lambda y. x) z w \rightarrow_{\beta} (\lambda y. z) w \rightarrow_{\beta} z
 \]
 これをde brujin indexで表わすと, $z, w \in \Lambda$に対して以下のようになる.
 \[
 (\lambda. \lambda. 1) z w \rightarrow_{\beta} (\lambda. 1)[z/0] w = (\lambda. 1[s (z, 1, 0)/1]) w \rightarrow_{\beta} 1[s (z, 1, 0)/1][w/0] = s (z, 1, 0)[w/0] = z
 \]
ここでの$\lambda. \lambda. 1$を$K$コンビネータという.
\end{ex}

ラムダ計算の基本的な性質として合流性がある.

\begin{thm}[Church Rosserの定理]\label{CR}
 $\beta$簡約$\rightarrow_{\beta}$は合流性を持つ.
\end{thm}

以下では定理\ref{CR}を示すために必要な準備を行う. まずparallel reductionと呼ばれる2項関係を定義する.

\begin{defn}
 parallel reduction$\rightarrow_{p}$とは$\Lambda$の上の$2$項関係であり, 以下の条件を満たす最小の関係のことをいう.
 \begin{description}
  \item[P1] 任意の$x, y \in V$について, $x = y$ならば$x \rightarrow_{p} y$
  \item[P2] 任意の$t, s \in \Lambda$について, $t \rightarrow_{p} s$ならば$\lambda. t \rightarrow_{p} \lambda. s$
  \item[P3] 任意の$t_1, t_2, s_1, s_2 \in \Lambda$について,
        $t_1 \rightarrow_{p} s_1$かつ$t_2 \rightarrow_{p} s_2$ならば$t_1 t_2 \rightarrow_{p} s_1 s_2$
  \item[P4] 任意の$t_1, t_2, s_1, s_2 \in \Lambda$について,
        $t_1 \rightarrow_{p} s_1$かつ$t_2 \rightarrow_{p} s_2$ならば$(\lambda. t_1) t_2 \rightarrow_{p} s_1[s_2/0]$
 \end{description}
\end{defn}

\begin{lem}\label{spp}
 代入はparallel reduction$\rightarrow_{p}$を保つ. すなわち$t_1, t_2, s_1, s_2 \in \Lambda$, $u \in V$に対して, $t_1 \rightarrow_{p} t_2$かつ$s_1 \rightarrow_{p} s_2$ならば$s_1[t_1/u] \rightarrow_{p} s_2[t_2/u]$.
\end{lem}

\begin{proof}
 $\rightarrow_{p}$の定義による帰納法を用いる. P4%TODO: fix
のみが非自明なので, その場合のみを示す.
 すなわち$p, q, t_1, t_2, s_1, s_2 \in \Lambda$に対して,
$p \rightarrow_{p} q$, $t_1 \rightarrow_{p} s_1$かつ $t_2 \rightarrow_{p} s_2$ならば$((\lambda. t_1) t_2) [p / s] \rightarrow_{p} (s_1 [s_2 / 0] [q / s])$を示す.
 
 代入を交換すると, 以下のような等式が成立する
 \[
 s_1 [s_2 / 0][q / s] = s_1 [s(q, 1, 0) / (s + 1)][s_2 [q / s] / 0]
 \]
 ことに注意する. 上の等式により, $t_1$に対しての帰納法の仮定を適用することができて, 主張が成立する.
\end{proof}

parallel reductionと$\beta$簡約の関係は以下のようになる.

\begin{lem}
 $\rightarrow_{\beta} \subset \rightarrow_{p} \subset \overset{*}{\rightarrow_{\beta}}$
\end{lem}

\begin{proof}
 前半は項のサイズによる帰納法を用いて示される. 後半は$\rightarrow_{p}$の定義による帰納法を用いて示される.
\end{proof}

\begin{cor}
 $\overset{*}{\rightarrow_{p}} = \overset{*}{\rightarrow_{\beta}}$
\end{cor}

上記の系と推移反射閉包を取る操作が合流性を保つことに注意すると, 定理\ref{CR}の言い換えが得られる.

\begin{thm}[Church Rosserの定理]\label{CR'}
 $\rightarrow_{p}$は合流性を持つ.
\end{thm}

\begin{proof}
 証明は\cite{高橋正子1991計算論}の71ページに従った.
任意の$M \in \Lambda$に対して, ある$N_M \in \Lambda$が存在し, 任意の$L \in \Lambda$に対して, $M \rightarrow_p L$ならば$L \rightarrow_p N_M$という主張を証明する.

$M \in \Lambda$を固定する. この$M$に対する帰納法を用いて主張を示す.

$M \in V$の場合, $N_M := M$とすれば, 任意の$L \in \Lambda$に対して, P1でのみ$M \rightarrow_p L$が与えられるため, このとき$L = M$. すなわち, $L \rightarrow_p N_M$.

$M = \lambda. M_0$となる$M_0 \in \Lambda$が存在する場合は, $N_M := \lambda. N_{M_0}$とするとよい.
実際, 任意の$L \in \Lambda$に対して, $M \rightarrow_p L$と仮定する.
このとき, ある$L_0 \in \Lambda$が存在して, $L = \lambda. L_0$, $M_0 \rightarrow_p L_0$となる.
なぜならば, $M \rightarrow_p L$となるときには, P2で$M \rightarrow_p L$が構成されることしか有り得ないためである.
すると帰納法の仮定より$L_0 \rightarrow_p N_{M_0}$, 再びP2により$L \rightarrow_p N_M$.

$M = (\lambda. M_0) M_1$となる$M_0, M_1 \in \Lambda$が存在する場合, $N_M := N_{M_0}[N_{M_1}/0]$とするとよい. 実際, 任意の$L \in \Lambda$に対して, $M \rightarrow_p L$と仮定する. このとき, $M \rightarrow_p L$の与えられ方はP3あるいはP4のいずれかとなる.
 
 P3のとき$L = (\lambda. L_0) L_1$となる$L_0, L_1 \in \Lambda$が存在し, $M_0 \rightarrow_p L_0$かつ$M_1 \rightarrow_p L_1$でなければならない. 帰納法の仮定により$L_0 \rightarrow_p N_{M_0}$かつ$L_1 \rightarrow_p N_{M_1}$となっているため, P4により$(\lambda. L_0) L_1 \rightarrow_p N_{M_0} [N_{M_1}/0]$が得られる.
 
 P4のとき$L = L_0 [L_1/0]$となる$L_0, L_1 \in \Lambda$が存在し, $M_0 \rightarrow_p L_0$かつ$M_1 \rightarrow_p L_1$でなければならない. 帰納法の仮定により$L_0 \rightarrow_p N_{M_0}$かつ$L_1 \rightarrow_p N_{M_1}$となっているため, 補題\ref{spp}より$L_0 [L_1/0] \rightarrow_p N_{M_0} [N_{M_1}/0]$が得られる.
 
最後に$M = M_0 M_1$となる$M_0, M_1 \in \Lambda$が存在する場合, $N_M := N_{M_0} N_{M_1}$とするとよい. これも自明である.
\end{proof}

\section{Calculus of Construction}
この節ではCoCの定義および停止性の証明を与える. 型なしラムダ計算における議論とは平行していることに注意されたい.
型なしの場合におけるラムダ項に対応するものとしてpseudotermと呼ばれるものを定義する.

\begin{defn}
 pseudoterm $p$とは, 以下のBNFで定義される木である. $v$は自然数を意味している.
 \[
 p := v | \lambda p. p | \Pi p. p | p p | \star | \square
 \]
 また, $P$をpseudotermの集合とする.
\end{defn}

\begin{defn}
 $\beta$簡約$\rightarrow_{\beta}$とは$P$の上の$2$項関係であり, 以下の条件を満たす最小の関係のことをいう.
 \begin{itemize}
  \item 任意の$M, T, N \in P$について, $(\lambda T. M) N \rightarrow_\beta M[N/0]$
  \item 任意の$N_1, N_2, M \in P$について, $N_1 \rightarrow_\beta N_2$ならば$\lambda M. N_1 \rightarrow_\beta \lambda M. N_2$かつ$\lambda N_1. M \rightarrow_\beta \lambda N_2. M$
  \item 任意の$N_1, N_2, M \in P$について, $N_1 \rightarrow_\beta N_2$ならば$\Pi M. N_1 \rightarrow_\beta \Pi M. N_2$かつ$\Pi N_1. M \rightarrow_\beta \Pi N_2. M$
  \item 任意の$N_1, N_2, M \in P$について, $N_1 \rightarrow_\beta N_2$ならば$M N_1 \rightarrow_\beta M N_2$かつ$N_1 M \rightarrow_\beta N_2 M$
 \end{itemize}
\end{defn}

 ただし, ここでもあいまいさを回避するためにde brujin indexによって代入を定義する.

\begin{defn}[de brujin index]
 写像 $s \colon P \times \mathbb{N} \times \mathbb{N} \rightarrow P$を任意のpseudoterm $M$, 自然数$n, c$に対して, 以下のように帰納的に定義する.

\[
  s (M, n, c) := \begin{cases}
    M & (M \in \mathbb{N} \text{かつ} M < c \text{ 又は } M = \star, \square) \\
    M + n & (M \in \mathbb{N} \text{かつ} M \geq c) \\
    \lambda s (N_1, n, c). s (N_2, n, c + 1) & (M = \lambda N_1. N_2) \\
    \Pi s (N_1, n, c). s (N_2, n, c + 1) & (M = \Pi N_1. N_2) \\
    s (N_1, n, c) s (N_2, n, c) & (M = N_1 N_2)
  \end{cases}
\]
 pseudoterm $M, N \in P$, 変数$v \in \mathbb{N}$に対する代入を以下のように帰納的に定義する.

\[
 M[N/v] := \begin{cases}
    N & (M = v) \\
    M - 1 & (M \in \mathbb{N} \text{かつ} v < M) \\
    M & (M \in \mathbb{N} \text{かつ} v \geq M \text{ 又は } M = \star, \square) \\
    \lambda M_1 [s (N, 1, 0)/v]. M_2 [s (N, 1, 0)/(v + 1)] & (M = \lambda M_1. M_2) \\
    \Pi M_1 [s (N, 1, 0)/v]. M_2 [s (N, 1, 0)/(v + 1)] & (M = \Pi M_1. M_2) \\
    N_1[N/v] N_2[N/v] & (M = N_1 N_2)
  \end{cases}
\]
\end{defn}

\ref{lambda}章とまったく同様の議論をおこなうことにより, CoCに対するChurch Rosserの定理を得ることができる.

\begin{thm}[Church Rosserの定理]
 $\beta$簡約$\rightarrow_{\beta}$は合流性を持つ.
\end{thm}

型づけ規則とよばれる自然演繹のシステムが存在し, それも含めての計算体系としてCoCは定義される.

\begin{defn}
 文脈$\Gamma$とは, 以下のBNFで定義される木である. $p$はpseudotermを意味する.
 \[
 \Gamma ::= \Gamma, p \mid
 \]
 % ただし, ここで述べている$v : p$とは型づけと呼ばれ, 型づけ規則とよばれるいくつかの公理から生成される木が存在することを意味している. 型づけ規則とは, 以下のように定義される.
 さらに, 型づけ規則とよばれるいくつかの公理から生成される木が定義される.
 
 pseudoterm $p, q \in P$が文脈$\Gamma$のもとで, $p : q$であるとは以下のいずれかの場合である.
 また, この状況を$\Gamma \vdash p : q$と書きあらわす.
 \begin{description}
  \item[Ax] $p = {\star}, q = {\square}$
  \item[Var] $p = 0$かつ, ある文脈$\Gamma'$が存在して$\Gamma = \Gamma', q$となる. またpseudoterm $t$が存在して$q = s(t, 1, 0)$となり, $\Gamma' \vdash t : \star$であるか$\Gamma' \vdash t : \square$となる. 
% さらに, $p$は$\Gamma'$に現れるどの変数とも異なっていなければならない.
  \item[Weak] ある文脈$\Gamma'$およびpseudoterm $r$が存在して$\Gamma = \Gamma', r$となり, $\Gamma' \vdash p : q$である. また, $\Gamma' \vdash r : \star$であるか$\Gamma' \vdash r : \square$となる. 
  \item[\Pi] あるpseudoterm $r, u$が存在して, $p = \Pi r. u$. かつ, $q = \star$あるいは$q = \square$. また$\Gamma \vdash r : \star$であるか$\Gamma \vdash r : \square$となる. さらに, $\Gamma, r \vdash u : q$が成立する.
  \item[\lambda] あるpseudoterm $r, u, m$が存在して, $p = \lambda r. m$, $q = \Pi r. u$. また, $\Gamma \vdash q : \star$であるか$\Gamma \vdash q : \square$となる. さらに, $\Gamma, r \vdash m : u$が成立する.
  \item[App] あるpseudoterm $n, m, u$が存在して, $p = m n$, $q = u[n/0]$. また, pseudoterm $t$が存在して$\Gamma \vdash m : \Pi t. u$かつ$\Gamma \vdash n : t$.
  \item[Conv] あるpseudoterm $t$が存在して, $t \overset{*}{\leftrightarrow}_\beta q$かつ$\Gamma \vdash p : t$. また$\Gamma \vdash q : \star$であるか$\Gamma \vdash q : \square$となる.
 \end{description}
 これらの定義の仕方は煩雑であるため, いくつかの規則によって生成される木が型づけとなることを暗黙の内に処理することが習慣となっている. そのような書き方による同値な定義を以下に述べる.
 \begin{description}
  \item[Ax] \[ \frac{}{\vdash {\star} {\colon} {\square}} \]
  \item[Var] \[ \frac{\Gamma \vdash t {\colon} {\star} \text{ or } {\square}}{\Gamma, t \vdash 0 {\colon} s(t, 1, 0)} \]
  \item[Weak] \[ \frac{\Gamma \vdash t {\colon} {\star} \text{ or } \square \quad \Gamma \vdash m {\colon} u}{\Gamma, t \vdash m {\colon} u} \]
  \item[\Pi] \[ \frac{\Gamma \vdash r {\colon} {\star} \text{ or } {\square} \quad \Gamma, r \vdash u {\colon} s}{\Gamma \vdash \Pi r. u {\colon} s} 
             \quad
             s = {\star} \text{ or } {\square} \]
  \item[\lambda] \[ \frac{\Gamma \vdash \Pi r. u {\colon} {\star} \text{ or } {\square} \quad \Gamma, r \vdash m : u}{\Gamma \vdash \lambda r. m {\colon} \Pi r. u} \]
  \item[App] \[ \frac{\Gamma \vdash m : \Pi t. u \quad \Gamma \vdash n : t}{\Gamma \vdash m n {\colon} u[n/0]} \]
  \item[Conv] \[\frac{\Gamma \vdash q {\colon} {\star} \text{ or } {\square} \quad \Gamma \vdash p {\colon} t \quad t \overset{*}{\leftrightarrow}_\beta q}{\Gamma \vdash p {\colon} q} \]
 \end{description}
 ここで現れる横に長い棒は含意を意味している. またpseudoterm $p, q$に対して, ある文脈$\Gamma$が存在して, $\Gamma \vdash p : q$となるとき, $\Gamma \vdash$を省略して$p : q$と書くことがある.
\end{defn}

\begin{rem}
 規則$\Pi$は四通り存在する.
\end{rem}

% \begin{rem}
%  上記の定義では文脈と型づけ規則が同時に定義されていることに注意されたい. したがって, 上記の定義は帰納的な定義になっており循環していない.
% \end{rem}

難しくない事実として, Subject Reductionと呼ばれるものがある.

\begin{thm}[Subject Reduction]
 文脈$\Gamma$, pseudoterm $A, B, C \in P$に対して, $\Gamma \vdash A \colon C$, $A \rightarrow_\beta B$のとき$\Gamma \vdash B \colon C$.
 % wrong statement
 % また$\Gamma \vdash C \colon A$, $A \rightarrow_\beta B$のとき$\Gamma \vdash C \colon B$.
\end{thm}
\begin{proof}
 型づけに対する帰納法を用いる.
 \begin{description}
  \item[Ax, Var] $\beta$簡約が存在しない.
  \item[Weak, Conv] 帰納法の仮定より直ちに従う.
  \item[\Pi] $\beta$簡約$A (= \Pi r. u) \rightarrow_\beta B$のうち, $B = \Pi r'. u, r \rightarrow_\beta r'$となる$r' \in P$がある場合が非自明である. この場合は規則$\Pi$, Weakおよび帰納法の仮定を用いれば主張が示せる.
  \item[\lambda] $\beta$簡約$A (= \lambda r. m) \rightarrow_\beta B$のうち, $B = \lambda r'. m, r \rightarrow_\beta r'$となる$r' \in P$がある場合が非自明である. この場合は規則Conv, $\lambda$, $\Pi$, Weakおよび帰納法の仮定を用いれば主張が示せる.
  \item[App] $\beta$簡約$A (= (\lambda t. m) n) \rightarrow_\beta B$のうち, $B = m[n/0]$となる$t, m \in P$がある場合が非自明である. この場合は規則$\lambda$, Weak, 帰納法の仮定および代入が型づけを保存することに注意すれば示せる.
 \end{description}
\end{proof}

\begin{ex}[恒等射]
 単純ではあるが普遍的なものとして恒等射があるが, CoCには恒等射を表現するのに十分な能力がある. (これは型理論にとっては自明ではない.)

 pseudoterm $x$を$x \colon {\square}$となる様なものとするとき, この上の恒等射としてふさわしい項は$\lambda x. 0 \colon \Pi x. s(x, 1, 0)$となる. 実際に$y \colon x$となる$y$をとると, 規則Appにより$(\lambda x. 0)y \colon s(x, 1, 0)[y/0] \rightarrow_\beta 0[y/0] \colon x = y \colon x$となることが確認される.

 また例えば$x = \star$に対して, $\lambda x. 0 \colon \Pi x. s(x, 1, 0)$が型づけ可能であることは以下の様に示される.
\begin{prooftree}
\AxiomC{$\vdash \star \colon \square$}
\AxiomC{$\vdash \star \colon \square$}
\AxiomC{$\vdash s(x, 1, 0) = \star \colon \square$}
\RightLabel{Weak}
\BinaryInfC{$x \vdash s(x, 1, 0) \colon \square$}
\RightLabel{$\Pi$}
\BinaryInfC{$\vdash \Pi x. s(x, 1, 0) \colon \square$}
\AxiomC{$\vdash \star \colon \square$}
\RightLabel{Var}
\UnaryInfC{$x \vdash 0 \colon s(x, 1, 0)$}
\RightLabel{$\lambda$}
\BinaryInfC{$\vdash \lambda x.0 \colon \Pi x. s(x, 1, 0)$}
\end{prooftree}
このような図式を証明木と言う.
\end{ex}

型つきの計算規則のもとでは, 型がつく項のみが考える対象となる. 型づけ可能な$P$上の$\beta$簡約に対しては停止性が成立する. 以下では停止性の証明に必要な準備を行っていく. 証明は\cite{geuvers1994short}, \cite{girard1989proofs}に従った.

簡単な帰納法により示される事実ではあるが, 重要なこととして以下の補題がある.

\begin{lem}[Classification]\label{clsfy}
 文脈$\Gamma$, pseudoterm $A, B \in P$に対して, $\Gamma \vdash A \colon B$となるのは以下の場合のいずれかに限る. また, それぞれの条件は重複しない.
 \begin{description}
  \item[Kind] $B = \square$
  \item[Constr] $\Gamma \vdash B {\colon} {\square}$
  \item[Obj] $\Gamma \vdash B {\colon} {\star}$
 \end{description}
\end{lem}
\begin{proof}
 型づけに対する帰納法を用いる. また, 証明を簡潔にするため, Convによる型づけの導出の不定性は無視する. 実際Subject Reductionにより, 分類を保ったまま導出からConvを消去することは可能である.
 \begin{description}
  \item[Ax] 自明にKindのみが成立する.
  \item[Var] 帰納法により$\square \colon \star$および$\square \colon \square$がおこらないことがわかるため, Kindはあり得ない. 型づけを生成する含意の前提に注目すると, 前提が$\Gamma \vdash t {\colon} {\star}$のとき, Weakを用いれば$\Gamma, t \vdash t {\colon} {\star}$が分かるので結論はObjであり, $\Gamma \vdash t {\colon} {\square}$のときはConstrとなる. 逆に結論に対してObjかつConstrであると仮定するとWeakで分解できて, $\Gamma \vdash t {\colon} {\star}$かつ$\Gamma \vdash t {\colon} {\square}$となるがこれは帰納法の仮定に反する.
  \item[Weak] 前提の二番目$\Gamma \vdash m {\colon} u$の分類が, 結論に一対一に対応することが直ちに分かる.
  \item[\Pi] 分類の一意性のみが非自明である. 前提の二番目$\Gamma, r \vdash u {\colon} s$に対して, Weakで分解できるので, 帰納法の仮定より主張が従う.
  \item[\lambda] 自明である.
  \item[App] 代入が型づけを保つ事実および, 規則$\Pi$が分類を保つことに注意すれば, 帰納法の仮定が反映される.
 \end{description}
\end{proof}

\begin{defn}
補題\ref{clsfy}を踏まえて, pseudotermの集合$P$をいくつかのクラスに分類する.
 \begin{itemize}
  \item $Kind := \{ A \mid \Gamma \text{が存在し} \Gamma \vdash A \colon {\square}\}$
  \item $Constr := \{ P \mid \Gamma, A \text{が存在し} \Gamma \vdash P \colon A \text{ かつ } \Gamma \vdash A \colon {\square}\}$
  \item $Obj := \{ P \mid \Gamma, A \text{が存在し} \Gamma \vdash P \colon A \text{ かつ } \Gamma \vdash A \colon {\star}\}$
 \end{itemize}
 補題\ref{clsfy}でみたようにこれらは互いにdisjointである.
\end{defn}

以下では$d \in \Lambda$として正規形となるものを一つ固定する. また$SN := \{ t \in \Lambda \mid t \text{は停止する}\}$とする.

\begin{defn}
 base terms $\mathcal{B}$を以下を満たす最小の集合として定義する.
 \begin{itemize}
  \item $\mathbb{N} \subset \mathcal{B}$かつ$d \in \mathcal{B}$
  \item $m \in \mathcal{B}$かつ$n \in SN$ならば, $mn \in \mathcal{B}$
 \end{itemize}
\end{defn}

\begin{defn}
 型なしラムダ計算のkey redexを以下を満たし, 最小になるように定義する.
 \begin{itemize}
  \item $m \in \Lambda$がredexであれば, $m$はkey redex.
  \item $m \in \Lambda$がkey redex $n \in \Lambda$を持てば, 任意の$p \in \Lambda$に対して, $m p$もkey redexとして$n$を持つ.
 \end{itemize}
ここで述べたredexとよばれる項は, 簡約がおこる部分木のことを指している. すなわち, $(\lambda. e) e' \quad (e, e' \in \Lambda)$となるような形の項のことを指している.
 
また, $m$に対するkey redexを$red_k(m)$と書く.
\end{defn}

\begin{defn}[Saturated set]\label{ss}
 型なしラムダ項の集合$X \subset \Lambda$がsaturatedであるとは, 以下のすべてを満たすものをいう.
 \begin{itemize}
  \item $X \subset SN$.
  \item $\mathcal{B} \subset X$.
  \item 任意の$m \in \Lambda$に対して, $red_k(m) \in X$かつ$m \in SN$ならば$m \in X$.
 \end{itemize}
 saturatedな集合の集合を$SAT$と書く.
\end{defn}

\begin{defn}
 解釈$\mathcal{V} : Kind \rightarrow Set$を以下のように帰納的に定義する. 
\[
  \mathcal{V} (A) := \begin{cases}
    SAT & (A = \star) \\
    \{ f \mid f : \mathcal{V}(B) \rightarrow \mathcal{V}(C) \} & (A = \Pi B. C \text{ かつ }B : \square) \\
    \mathcal{V} (C) & (A = \Pi B. C  \text{ かつ }B : \star)
  \end{cases}
\]
 $\mathcal{V}$の値域を$(SAT)^\star := \bigcup_{A \in Kind} \{\mathcal{V}(A)\}$のように書く.
\end{defn}
% 以下では$T := \{p \in P | t \in P\text {が存在し} t : p \text{または} p : t \}$とし, $T$の元をtermと呼ぶ. $T$については$\beta$簡約の上

\begin{lem}
 $SAT$は交差と含意で閉じる. すなわち以下を満たす.
 \begin{itemize}
  \item saturated set $X, Y$に対して, $X \cap Y$もsaturatedとなる.
  \item saturated set $X, Y$に対して, $\{ M \in \Lambda \mid \text{任意の} N \in X \text{に対して}, M N \in Y\}$もsaturatedとなる.
 \end{itemize}
 ここで現れた $\{ M \in \Lambda \mid \text{任意の} N \in X \text{に対して}, M N \in Y\}$を$X \rightarrow Y$とかく.
\end{lem}
\begin{proof}
 前半は自明である.
 後半をみるには定義\ref{ss}の帰納的ないいかえを考えるとよい. すなわち, 
 型なしラムダ項の集合$X \subset \Lambda$がsaturatedであるとは, 以下のすべてを満たすものをいう.
 \begin{itemize}
  \item $X \subset SN$.
  \item $n \in \mathbb{N}, q_1, \ldots, q_n \in SN, x \in \mathbb{N}$に対して, $x q_1 \ldots q_n \in X$.
  \item $n \in \mathbb{N}, q_1, \ldots, q_n \in SN$に対して, $d q_1 \ldots q_n \in X$.
  \item $n \in \mathbb{N}, q_1, \ldots, q_n, m, p \in SN$に対して, $m[p/0]q_1 \ldots q_n \in X$ならば$(\lambda m)p q_1 \ldots q_n \in X$.
 \end{itemize}
最後のいいかえについては, key redexの定義を考えればすぐにいえる.
さて, $n \in \mathbb{N}, q_1, \ldots, q_n, m, p \in SN$に対して, $m[p/0]q_1 \ldots q_n \in X \rightarrow Y$とする. このとき, $(\lambda m)p q_1 \ldots q_n \in X \rightarrow Y$を言いたい.
すなわち, 任意の$l \in X$に対して, $(\lambda m)p q_1 \ldots q_n l \in Y$となることを言いたい.
 $Y$はsaturated setであったので, $m[p/0] q_1 \ldots q_n l \in Y$であればよいが, 
 $m[p/0]q_1 \ldots q_n \in X \rightarrow Y$であったので主張が従う.
\end{proof}

\begin{thm}
 型づけ可能な$P$上の$\beta$簡約$\rightarrow_{\beta}|_T$は停止性を持つ.
\end{thm}
\begin{proof}
\end{proof}

%\section{論理体系としてのCoC}
% SN と CR -> type jugdement が decidable.
% domain のなかで証明列をながめると?
% CoCでないと扱えないようなStatementの例とは? : Proposition の合成?
% 3

\section{まとめ}
% 結論がよわいような...
% 1

\section{謝辞}

\bibliography{bib}
\bibliographystyle{junsrt}
\end{document}
