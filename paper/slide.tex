\documentclass[18pt]{beamer}
\usetheme{metropolis}
\usepackage{listings}
\usepackage{luatexja}
\usepackage{minted}
\usepackage{bussproofs}
\setbeamersize{text margin left=5mm, text margin right=5mm}
\begin{document}

\title{Calculus of Constructionの基本的性質}
\author{321701183 田中 一成}


\frame{\maketitle}

\begin{frame}{型理論とは}
 型とは
 \begin{itemize}
  \item プログラミング言語におけるデータの種類
  \item 写像の定義域と値域
  \item 所属関係が決定可能な集合(論)
  \item 主張に対しての証明
 \end{itemize}
\end{frame}

\begin{frame}[fragile]{プログラミング言語におけるデータの種類/写像の定義域と値域}
\begin{tabular}{c}
 \begin{minipage}{0.5\textwidth}
  \begin{minted}[mathescape, framesep=2mm]{c}
   int f(int x) {
       return x;
   }
  \end{minted}
 \end{minipage}
 \begin{minipage}{0.5\textwidth}
 \[
 f \colon \mathbb{N} \rightarrow \mathbb{N}
 \]
 \end{minipage}
\end{tabular}
\end{frame}

\begin{frame}[fragile]{所属関係が決定可能な集合(論)/主張に対しての証明}
 \[
 x \mapsto f (x) \in \{ g \mid g \colon \mathbb{N} \rightarrow \mathbb{N} \}
 \]
\begin{prooftree}
\AxiomC{$x \colon \mathbb{N} \vdash f(x) \colon \mathbb{N}$}
\UnaryInfC{$\vdash x \mapsto f(x) \colon \mathbb{N} \rightarrow \mathbb{N}$}
\end{prooftree}
\end{frame}

\begin{frame}{ラムダ計算}
 \begin{itemize}
  \item 最も簡潔な計算体系
  \item $t ::= v \mid \lambda v. t \mid t t$
  \item $(\lambda v. M) N \rightarrow_\beta M[N/v]$
  \item $N_1 \rightarrow_\beta N_2$ならば$(\lambda v. N_1) \rightarrow_\beta (\lambda v. N_2)$ \item $N_1 \rightarrow_\beta N_2$ならば$M N_1 \rightarrow_\beta M N_2$かつ$N_1 M \rightarrow_\beta N_2 M$
 \end{itemize}
\end{frame}

\begin{frame}{Calculus of Construction(CoC)}
 \begin{itemize}
  \item ラムダ項に対して型をつける
  \item $p ::= v \mid \lambda v \colon p. p \mid \Pi v \colon p. p \mid p p \mid \star \mid \square$
  \item $(\lambda v \colon T. M) N \rightarrow_\beta M[N/v]$
  \item $N_1 \rightarrow_\beta N_2$ならば$\lambda v \colon M. N_1 \rightarrow_\beta \lambda v \colon M. N_2$かつ$\lambda v \colon N_1. M \rightarrow_\beta \lambda v \colon N_2. M$
  \item $N_1 \rightarrow_\beta N_2$ならば$\Pi v \colon M. N_1 \rightarrow_\beta \Pi v \colon M. N_2$かつ$\Pi v \colon N_1. M \rightarrow_\beta \Pi v \colon N_2. M$
  \item $N_1 \rightarrow_\beta N_2$ならば$M N_1 \rightarrow_\beta M N_2$かつ$N_1 M \rightarrow_\beta N_2 M$
 \end{itemize}
\end{frame}

\begin{frame}[fragile]{型つけ規則}
\begin{tabular}{rl}
 \begin{minipage}{135pt}
 \begin{description}
  \item[\scriptsize{Ax}] $\vdash {\star} {\colon} {\square}$
  \item[\scriptsize{Var}] $\frac{\Gamma \vdash t {\colon} {\star} \text{ or } {\square}}{\Gamma, v \colon t \vdash v {\colon} t}$
  \item[\scriptsize{Weak}] $\frac{\Gamma \vdash t {\colon} {\star} \text{ or } \square \quad \Gamma \vdash m {\colon} u}{\Gamma, v \colon t \vdash m {\colon} u}$
  \item[\Pi] $\frac{\Gamma \vdash r {\colon} {\star} \text{ or } {\square} \quad \Gamma, v \colon r \vdash u {\colon} s}{\Gamma \vdash \Pi v \colon r. u {\colon} s}$
  \item $s = {\star} \text{ or } {\square}$
 \end{description}
 \end{minipage} &
 \begin{minipage}{130pt}
 \begin{description}
  \item[\lambda] $\frac{\Gamma \vdash \Pi v \colon r. u {\colon} {\star} \text{ or } {\square} \quad \Gamma, v \colon r \vdash m : u}{\Gamma \vdash \lambda v \colon r. m {\colon} \Pi v \colon r. u}$
  \item[\scriptsize{App}] $\frac{\Gamma \vdash m : \Pi v \colon t. u \quad \Gamma \vdash n : t}{\Gamma \vdash m n {\colon} u[n/v]}$
  \item[\scriptsize{Conv}] $\frac{\Gamma \vdash q {\colon} {\star} \text{ or } {\square} \quad \Gamma \vdash p {\colon} t \quad t \overset{*}{\leftrightarrow}_\beta q}{\Gamma \vdash p {\colon} q}$
  \item $\Gamma ::= \Gamma, v \colon p \mid$ 
 \end{description}
 \end{minipage} \\
\end{tabular}
\end{frame}

\end{document}