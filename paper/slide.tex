\documentclass[18pt]{beamer}
\usetheme{metropolis}
\usepackage{listings}
\usepackage{luatexja}
\usepackage{minted}
\usepackage{bussproofs}

\title{Calculus of Constructionの基本的性質}
\author{321701183 田中 一成}

\begin{document}

\frame{\maketitle}

\begin{frame}{型理論とは}
 型とは
 \begin{itemize}
  \item プログラミング言語におけるデータの種類
  \item 写像の定義域と値域
  \item 所属関係が決定可能な集合(論)
  \item 主張に対しての証明
 \end{itemize}
\end{frame}

\begin{frame}[fragile]{プログラミング言語におけるデータの種類/写像の定義域と値域}
\begin{tabular}{c}
 \begin{minipage}{0.5\textwidth}
  \begin{minted}[mathescape, framesep=2mm]{c}
   int f(int x) {
       return x;
   }
  \end{minted}
 \end{minipage}
 \begin{minipage}{0.5\textwidth}
 \[
 f \colon \mathbb{N} \rightarrow \mathbb{N}
 \]
 \end{minipage}
\end{tabular}
\end{frame}


\begin{frame}[fragile]{所属関係が決定可能な集合(論)/主張に対しての証明}
 \[
 x \mapsto f (x) \in \{ g \mid g \colon \mathbb{N} \rightarrow \mathbb{N} \}
 \]
\begin{prooftree}
\AxiomC{$x \colon \mathbb{N} \vdash f(x) \colon \mathbb{N}$}
\UnaryInfC{$\vdash x \mapsto f(x) \colon \mathbb{N} \rightarrow \mathbb{N}$}
\end{prooftree}
\end{frame}

\end{document}