\documentclass[18pt]{beamer}
\usetheme{metropolis}
\usepackage{listings}
\usepackage{luatexja}
\usepackage{minted}
\usepackage{bussproofs}
\usepackage{amsmath,amssymb,amsthm}
\usepackage{stmaryrd}
\usepackage{tikz}
\usetikzlibrary{matrix, positioning, cd}
\DeclareMathOperator{\SN}{SN}
\DeclareMathOperator{\Term}{Term}
\DeclareMathOperator{\Obj}{Obj}
\DeclareMathOperator{\Constr}{Constr}
\DeclareMathOperator{\Type}{Type}
\DeclareMathOperator{\Kind}{Kind}
\DeclareMathOperator{\SAT}{SAT}
\DeclareMathOperator{\Set}{Set}
\DeclareMathOperator{\GTerm}{\Gamma-Term}
\DeclareMathOperator{\GObj}{\Gamma-Obj}
\DeclareMathOperator{\GConstr}{\Gamma-Constr}
\DeclareMathOperator{\GType}{\Gamma-Type}
\DeclareMathOperator{\GKind}{\Gamma-Kind}
\DeclareMathOperator{\K}{K}
\DeclareMathOperator{\red}{red}
\newcommand{\iprt}[2]{\llbracket {#1} \rrbracket_ {#2}}

\setbeamersize{text margin left=5mm, text margin right=5mm}
\begin{document}

\theoremstyle{definition}
\newtheorem{defn}{定義}
\newtheorem{thm}{定理}
\newtheorem{lem}{補題}
\newtheorem{rem}{注意}
\newtheorem{cor}{系}
\newtheorem{ex}{例}
\renewcommand{\proofname}{\bf{証明}}

\title{Calculus of Construction(CoC)の基本的性質}
\author{321701183 田中 一成}

\frame{\maketitle}

\begin{frame}{歴史}
 \begin{itemize}
  \item Russell $\quad$ 
  \item Church $\quad$ 
  \item Martin lof
  \item Coquand $\quad$ CoCを考案
  \item Coq $\quad$ CoCを援用した定理証明支援系
 \end{itemize}
\end{frame}

\begin{frame}{型とは}
 型とは
 \begin{itemize}
  \item プログラミング言語におけるデータの種類
  \item 数学の写像の定義域と値域
  \item 何かについての主張
 \end{itemize}
\end{frame}

\begin{frame}[fragile]{プログラミング言語におけるデータの種類/写像の定義域と値域}
\begin{tabular}{c}
 \begin{minipage}{0.5\textwidth}
  \begin{minted}[mathescape, framesep=2mm]{c}
   int f(int x) {
       return x;
   }
  \end{minted}
 \end{minipage}
 \begin{minipage}{0.5\textwidth}
 \[
 f \colon \mathbb{N} \rightarrow \mathbb{N}
 \]
 \end{minipage}
\end{tabular}
\end{frame}

\begin{frame}[fragile]{何かについての主張}
\begin{tabular}{cc}
 \begin{minipage}{0.5\textwidth}
 \[
 x \mapsto f (x) \in \{ g \mid g \colon \mathbb{N} \rightarrow \mathbb{N} \}
 \]
 \end{minipage} &
 \begin{minipage}{0.5\textwidth}
\begin{prooftree}
\AxiomC{$x \colon \mathbb{N} \vdash f(x) \colon \mathbb{N}$}
\UnaryInfC{$\vdash x \mapsto f(x) \colon \underbrace{\mathbb{N} \rightarrow \mathbb{N}}_{\text{何かについての主張}}$}
\end{prooftree}
 \end{minipage} //
\end{tabular}
\end{frame}

% 関数であることをいう
\begin{frame}{ラムダ計算}
 \begin{itemize}
  \item 簡潔な計算体系
  \item $t ::= v \mid \lambda v. t \mid t t$
  \item $(\lambda v. M) N \rightarrow_\beta M[N/v]$
  \item $N_1 \rightarrow_\beta N_2$ならば$(\lambda v. N_1) \rightarrow_\beta (\lambda v. N_2)$ 
  \item $N_1 \rightarrow_\beta N_2$ならば$M N_1 \rightarrow_\beta M N_2$かつ$N_1 M \rightarrow_\beta N_2 M$
 \end{itemize}
\end{frame}

\begin{frame}{ラムダ計算}
 \begin{itemize}
  \item おおよそこう読める
  \item $\lambda v. M \simeq v \mapsto M$
  \item $(\lambda v. M) N \simeq (v \mapsto M) (N)$
  \item 例えば $\lambda v. v + 1 \simeq v \mapsto v + 1$
 \end{itemize}
\end{frame}

\begin{frame}{例 $\colon$ 自然数}
 \begin{itemize}
  \item Church Encodingとよばれる自然数の表現が知られている
  \item これはラムダ計算の上で与えられる
 \end{itemize}
\end{frame}

\begin{frame}{例 $\colon$ 自然数}
 \begin{align*}
 n & := \lambda f. \lambda x. \underbrace{f (\cdots (f}_{n} x)) \\
 n + m & := \lambda f. \lambda x. (m f) (n f x) \\
       & \rightarrow_\beta \lambda f. \lambda x. (\lambda y. \underbrace{g (\cdots (g}_{m} y))[g/f]) (n f x) \\
       & = \lambda f. \lambda x. (\lambda y. \underbrace{f (\cdots (f}_{m} y))) (n f x) \\
       & \rightarrow_\beta \lambda f. \lambda x. (\underbrace{f (\cdots (f}_{m} y)))[y/(n f x)] \\
       & = \lambda f. \lambda x. \underbrace{f (\cdots (f}_{m} (n f x))) \\
       & \rightarrow_\beta \lambda f. \lambda x. \underbrace{f (\cdots (f}_{m} (\underbrace{f (\cdots (f}_{n} x)))))
 \end{align*}
\end{frame}

\begin{frame}{Calculus of Construction(CoC)}
 \begin{itemize}
  \item ラムダ項に対して型をつける
  \item $p ::= v \mid \lambda v \colon p. p \mid \Pi v \colon p. p \mid p p \mid \star \mid \square$
  \item $(\lambda v \colon T. M) N \rightarrow_\beta M[N/v]$
  \item $N_1 \rightarrow_\beta N_2$ならば$\lambda v \colon M. N_1 \rightarrow_\beta \lambda v \colon M. N_2$かつ$\lambda v \colon N_1. M \rightarrow_\beta \lambda v \colon N_2. M$
  \item $N_1 \rightarrow_\beta N_2$ならば$\Pi v \colon M. N_1 \rightarrow_\beta \Pi v \colon M. N_2$かつ$\Pi v \colon N_1. M \rightarrow_\beta \Pi v \colon N_2. M$
  \item $N_1 \rightarrow_\beta N_2$ならば$M N_1 \rightarrow_\beta M N_2$かつ$N_1 M \rightarrow_\beta N_2 M$
 \end{itemize}
\end{frame}

% 論理に対応させる
\begin{frame}[fragile]{型つけ規則}
\begin{tabular}{cccc}
 $\vdash {\star} {\colon} {\square}$ & (Ax) & $\frac{\Gamma \vdash t {\colon} {\star} \text{ or } {\square}}{\Gamma, v \colon t \vdash v {\colon} t}$ & (Var) \\[30pt]
 $\frac{\Gamma \vdash t {\colon} {\star} \text{ or } \square \quad \Gamma \vdash m {\colon} u}{\Gamma, v \colon t \vdash m {\colon} u}$ & (Weak) & $\frac{\Gamma \vdash r {\colon} {\star} \text{ or } {\square} \quad \Gamma, v \colon r \vdash u {\colon} s}{\Gamma \vdash \Pi v \colon r. u {\colon} s}$ & ($\Pi$) \\[30pt]
 $\frac{\Gamma \vdash \Pi v \colon r. u {\colon} {\star} \text{ or } {\square} \quad \Gamma, v \colon r \vdash m : u}{\Gamma \vdash \lambda v \colon r. m {\colon} \Pi v \colon r. u}$ & ($\lambda$) & $\frac{\Gamma \vdash m : \Pi v \colon t. u \quad \Gamma \vdash n : t}{\Gamma \vdash m n {\colon} u[n/v]}$ & (App) \\[30pt]
 $\frac{\Gamma \vdash q {\colon} {\star} \text{ or } {\square} \quad \Gamma \vdash p {\colon} t \quad t \overset{*}{\leftrightarrow}_\beta q}{\Gamma \vdash p {\colon} q}$ & (Conv) & $s = {\star} \text{ or } {\square}$ & \\[30pt]
& & $\Gamma ::= \Gamma, v \colon p \mid$ & \\
\end{tabular}
\end{frame}

\begin{frame}{例 $\colon$ 自然数}
 \begin{itemize}
  \item 
 \end{itemize}
\end{frame}

% \begin{frame}{Church Rosser}
% \begin{thm}[Church Rosserの定理]
%  $\beta$簡約$\rightarrow_{\beta}$は合流性を持つ.
% \end{thm}

% \begin{minipage}
%  % https://tikzcd.yichuanshen.de/#N4Igdg9gJgpgziAXAbVABwnAlgFyxMJZABgBpiBdUkANwEMAbAVxiRAFkQBfU9TXfIRQBGclVqMWbAHLdeIDNjwEiZYePrNWiEAAU5fJYKKj11TVJ3sA5N3EwoAc3hFQAMwBOEALZIyIHAgkURAGOgAjGAZdfmUhUJg3HAMQTx8-akCkACZqMMjo2OMdDyxHAAtknncvX0QQrMQAZjysMG0QKDo4cocQPIiomKMVHQZE5PNJDoAdGZgADyw4HARq1NqczKDm1va2Lp6+gYLhgVGQUorJiS02OcXl1bsuIA
% \begin{tikzcd}
% M \arrow[r] \arrow[d]           & N \arrow[d, "\exists", dashed] \\
% P \arrow[r, "\exists"', dashed] & M'                            
% \end{tikzcd}
% \end{minipage}
% \end{frame}

\begin{frame}{Church Rosser}
\begin{thm}[Church Rosserの定理]
 $\beta$簡約$\rightarrow_{\beta}$は合流性を持つ.
\end{thm}
 
\begin{proof}
 \begin{description}
  \item $v_1 = v_2$ならば$v_1 \rightarrow_{p} v_2$
  \item $t \rightarrow_{p} s$ならば$\lambda v \colon T. t \rightarrow_{p} \lambda v \colon T. s$
  \item $t \rightarrow_{p} s$ならば$\Pi v \colon T. t \rightarrow_{p} \Pi v \colon T. s$
  \item $t_1 \rightarrow_{p} s_1$かつ$t_2 \rightarrow_{p} s_2$ならば$t_1 t_2 \rightarrow_{p} s_1 s_2$
  \item $t_1 \rightarrow_{p} s_1$かつ$t_2 \rightarrow_{p} s_2$ならば$(\lambda v \colon T. t_1) t_2 \rightarrow_{p} s_1[s_2/v]$
 \end{description}
\end{proof}
\end{frame}

\begin{frame}{Subject Reduction}
\begin{thm}[Subject Reduction]
 $\Gamma \vdash A \colon C$, $A \rightarrow_\beta B$のとき$\Gamma \vdash B \colon C$
\end{thm}
\begin{proof}
 型づけに対する帰納法を用いる.
\end{proof}
\end{frame}

\end{document}